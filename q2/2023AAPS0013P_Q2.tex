\documentclass{AnantDocumentation}

\title{ECE F343: Communication Networks \\ \large{Assignment 1}}
\author{Pranav Chandra N. V. \\ 2023AAPS0013P}

\begin{document}
\maketitle

\begin{enumerate}
	\item \textbf{What protocols are listed in the Wireshark “protocol” column in your trace file?
		      Make a list of such protocols, identify the layer to which they belong, and briefly explain (in 1-2 lines) the function of each protocol.
	      }

	      The protocols listed under the "protocol" column are as follows:
	      \begin{enumerate}
		      \item \textbf{ARP - Layer 2/3 (Network Interface / Network)} - Maps an IPv4 address to a MAC address on a local network.

		      \item \textbf{mDNS - Layer 7 (Application)} - Resolves hostnames locally using multicast without a DNS server.

		      \item \textbf{IGMPv2 - Layer 3 (Network)} - Manages IPv4 multicast group membership.

		      \item \textbf{ICMPv6 - Layer 3 (Network)} - Provides IPv6 error reporting and neighbor discovery.

		      \item \textbf{TCP - Layer 4 (Transport)} - Provides reliable, connection-oriented data transmission.

		      \item \textbf{SSDP - Layer 7 (Application)} - Discovers devices and services on a local network.

		      \item \textbf{LLC - Layer 2 (Data Link)} - Identifies upper-layer protocols within Ethernet frames.

		      \item \textbf{VRRP - Layer 3 (Network)} - Provides gateway redundancy using a virtual IP address.

		      \item \textbf{DHCP - Layer 7 (Application)} - Automatically assigns IP configuration to clients.

		      \item \textbf{TLSv1.2 - Layer 6/7 (Presentation/Application)} - Encrypts and secures application data.

		      \item \textbf{DNS - Layer 7 (Application)} - Translates domain names into IP addresses.

		      \item \textbf{UDP - Layer 4 (Transport)} - Provides fast, connectionless data transmission.

		      \item \textbf{IGMPv3 - Layer 3 (Network)} - Supports source-specific IPv4 multicast membership.

		      \item \textbf{NBNS - Layer 7 (Application)} - Resolves NetBIOS names to IP addresses.

		      \item \textbf{TLSv1.3 - Layer 6/7 (Presentation/Application)} - Provides faster and more secure encryption than TLS 1.2.

		      \item \textbf{HTTP - Layer 7 (Application)} - Transfers web content using a request-response model.

		      \item \textbf{BROWSER - Layer 7 (Application)} - Maintains shared resource lists in Windows networks.
	      \end{enumerate}

	\item \textbf{Read about HTTTP protocol and its working (Reference: Section 2.1, 8.1 in Garcia).
		      Now in your experiment, determine how long did it take from when the HTTP GET message
		      was sent until the HTTP OK reply was received? (By default, the value of the Time column in
		      the packet-listing window is the amount of time, in seconds, since Wireshark tracing began.
		      (If you want to display the Time field in time-of-day format, select the Wireshark View pull
		      down menu, then select Time Display Format, then select Time-of-day.)}

	      The HTTPS GET packet was triggered at $t = 19.08404851$ and the OK packet was recieved at $t = 19.830304061$, meaning it took $0.746255551000001$ seconds for this communicaiton.

	\item \textbf{What is the Internet address of the gaia.cs.umass.edu (also known as www-
		      net.cs.umass.edu)? What is the Internet address of the computer that sent the HTTP GET
		      message (i.e., your computer)?}

	      Since the website is a https secured website, we use the dns filter, rather than the http filter on the wireshark protocols list. The source and destination are as follows
	      \begin{itemize}
		      \item \textbf{Source: } 172.24.2.76
		      \item
	      \end{itemize}

\end{enumerate}



\end{document}

